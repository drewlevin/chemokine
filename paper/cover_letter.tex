\documentclass[10pt]{article}

% amsmath package, useful for mathematical formulas
\usepackage{amsmath}
% amssymb package, useful for mathematical symbols
\usepackage{amssymb}

% graphicx package, useful for including eps and pdf graphics
% include graphics with the command \includegraphics
\usepackage{graphicx}

% cite package, to clean up citations in the main text. Do not remove.
\usepackage{cite}

\usepackage{color} 

% Use doublespacing - comment out for single spacing
\usepackage{setspace} 
%\doublespacing

% Text layout
\topmargin 0.0cm
\oddsidemargin 0.5cm
\evensidemargin 0.5cm
\textwidth 16cm 
\textheight 21cm

% Bold the 'Figure #' in the caption and separate it with a period
% Captions will be left justified
\usepackage[labelfont=bf,labelsep=period,justification=raggedright]{caption}

% Use the PLoS provided bibtex style
\bibliographystyle{plos2009}

% Remove brackets from numbering in List of References
\makeatletter
\renewcommand{\@biblabel}[1]{\quad#1.}
\makeatother


% Leave date blank
\date{}

\pagestyle{myheadings}
%% ** EDIT HERE **

\usepackage{multirow}

%% ** EDIT HERE **
%% PLEASE INCLUDE ALL MACROS BELOW

% figure files reside in the figures/ directory
\graphicspath{
{figures/}
}

%% END MACROS SECTION

\begin{document}

\pagenumbering{gobble}

\begin{flushleft}Dear Dr. Denise Kirschner \\
Co-Editor-in-Chief \\
Journal of Theoretical Biology, \end{flushleft} 

%We submit this manuscript entitled "A spatial model of the efficiency of T cell search in the influenza-infected lung" for consideration of publication in Royal Society Interface.  Our paper implements an agent-based model of bronchial influenza infection and the subsequent T cell response using new empirical values of chemokine secretion.  The paper shows that chemokine levels have a limited yet important role in directing T cell migration and that the spatial effects of chemokine diffusion can actually hinder T cell search in rapidly spreading viruses.  Our sensitivity analysis reveals the individual contributions of the model parameters on the course of the infection.  Finally, we calculate a maximum threshold for T cell search time, beyond which leads to uncontrolled infection.  We feel this work will be of interest to investigators modeling the immune response for the ultimate purpose of improving vaccine design.

We are pleased to submit the revised manuscript, ``A spatial model of the efficiency of T cell search in the influenza-infected lung" for JTB's consideration. This is a resubmission of manuscript number JTB-D-15-00501, received by us on August 12, 2015 with an invitation for resubmission within 60 days. \\

The reviewer comments were extremely helpful, and we have addressed each of their suggestions in the attached Response to Reviewers.   To summarize the more important changes, we performed the recommended PRCC sensitivity analysis for one viral strain. Because it confirmed the results of our original sensitivity analysis, we conducted this computationally intensive analysis for only this strain.  Beyond the PRCC runs, we doubled the number of model runs using our standard parameters.  This resulted in a very small decrease in the variance of the results and did not affect the mean.  This gives us additional confidence in the validity of the reported results.  Next, we reorganized the Methods and Model section to make the rationale for our parameter choices more clear.  We note that the PRCC sensitivity analysis confirms that many of these parameters do not have a strong effect on the model output.  Finally, we corrected the typographical errors pointed out by the referees.   \\

We feel that the paper is much improved with these changes, and we appreciate your patience during the revision process.

\begin{flushleft}Sincerely, \end{flushleft} 

\noindent Drew Levin \\
University of New Mexico Department of Computer Science \\
MSC01 1130 \\
1 University of New Mexico \\
Albuquerque, NM 87131 \\
drew@cs.unm.edu

\end{document}
