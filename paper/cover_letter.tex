\documentclass[10pt]{article}

% amsmath package, useful for mathematical formulas
\usepackage{amsmath}
% amssymb package, useful for mathematical symbols
\usepackage{amssymb}

% graphicx package, useful for including eps and pdf graphics
% include graphics with the command \includegraphics
\usepackage{graphicx}

% cite package, to clean up citations in the main text. Do not remove.
\usepackage{cite}

\usepackage{color} 

% Use doublespacing - comment out for single spacing
\usepackage{setspace} 
%\doublespacing

% Text layout
\topmargin 0.0cm
\oddsidemargin 0.5cm
\evensidemargin 0.5cm
\textwidth 16cm 
\textheight 21cm

% Bold the 'Figure #' in the caption and separate it with a period
% Captions will be left justified
\usepackage[labelfont=bf,labelsep=period,justification=raggedright]{caption}

% Use the PLoS provided bibtex style
\bibliographystyle{plos2009}

% Remove brackets from numbering in List of References
\makeatletter
\renewcommand{\@biblabel}[1]{\quad#1.}
\makeatother


% Leave date blank
\date{}

\pagestyle{myheadings}
%% ** EDIT HERE **

\usepackage{multirow}

%% ** EDIT HERE **
%% PLEASE INCLUDE ALL MACROS BELOW

% figure files reside in the figures/ directory
\graphicspath{
{figures/}
}

%% END MACROS SECTION

\begin{document}

\pagenumbering{gobble}

\begin{flushleft}Dear Editors, \end{flushleft} 

We submit this manuscript entitled "A spatial model of the efficiency of T cell search in the influenza-infected lung" for consideration of publication in Royal Society Interface.  Our paper implements an agent-based model of bronchial influenza infection and the subsequent T cell response using new empirical values of chemokine secretion.  The paper shows that chemokine levels have a limited yet important role in directing T cell migration and that the spatial effects of chemokine diffusion can actually hinder T cell search in rapidly spreading viruses.  Our sensitivity analysis reveals the individual contributions of the model parameters on the course of the infection.  Finally, we calculate a maximum threshold for T cell search time, beyond which leads to uncontrolled infection.  We feel this work will be of interest to investigators modeling the immune response for the ultimate purpose of improving vaccine design.
 \\

We recommend the following reviewers for their expertise in influenza and/or immunological modeling:
\begin{enumerate}
\item Judy Day - University of Tennessee 
\item Grant Lythe - Leeds University
\item Carmen Molina-Paris - Leeds University
\item Johannes Textor - University of Utrecht
\end{enumerate}

We suggest that you excuse the following people from review due to involvement in and discussion of earlier iterations of this work:
\begin{enumerate}
\item Alan Perelson - Los Alamos National Laboratories 
\item Catherine Beauchemin - Ryerson University
\item Rustom Antia - Emory University
\end{enumerate}

This work has not been published elsewhere, and is complementary to a previous publication from our group (H Mitchell, D Levin, et al. J Virol 2011;85(2):1125-1135). \\


% This work is a resubmission of a rejected PLOS CB manuscript (PCOMPBIOL-D-12-01516R1).

All authors have read and approved the revised manuscript and report no conflicts of interest. 

\begin{flushleft}Respectfully yours,\end{flushleft} 

\noindent Drew Levin, et. al. \\
University of New Mexico Department of Computer Science \\
MSC01 1130 \\
1 University of New Mexico \\
Albuquerque, NM 87131 \\
drew@cs.unm.edu

\end{document}
